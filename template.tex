\documentclass{article}
\usepackage[utf8]{inputenc}
\usepackage{amsfonts,latexsym,amsthm,amssymb,amsmath,amscd,euscript}
\usepackage{mathtools}
\usepackage{framed}
% Descomentar fullpage cuando se quiera utilizar menos margen horizontal
%\usepackage{fullpage}
\usepackage{hyperref}
    \hypersetup{colorlinks=true,citecolor=blue,urlcolor =black,linkbordercolor={1 0 0}}

\newenvironment{statement}[1]{\smallskip\noindent\color[rgb]{1.00,0.00,0.50} {\bf #1.}}{}
\allowdisplaybreaks[1]

% Comandos para teoremas, definiciones, ejemplos, lemas, etc. para sus respectivos body types.
\renewcommand*{\proofname}{Prueba}
\renewcommand{\contentsname}{Contenido}

\newtheorem{theorem}{Teorema}
\newtheorem*{proposition}{Proposici\'on}
\newtheorem{lemma}[theorem]{Lema}
\newtheorem{corollary}[theorem]{Corolario}
\newtheorem{conjecture}[theorem]{Conjetura}
\newtheorem*{postulate}{Postulado}
\theoremstyle{definition}
\newtheorem{defn}[theorem]{Definici\'on}
\newtheorem{example}[theorem]{Ejemplo}

\theoremstyle{remark}
\newtheorem*{remark}{Observaci\'on}
\newtheorem*{notation}{Notaci\'on}
\newtheorem*{note}{Nota}

% Define tus comandos para hacer la vida más fácil.
\newcommand{\BR}{\mathbb R}
\newcommand{\BC}{\mathbb C}
\newcommand{\BF}{\mathbb F}
\newcommand{\BQ}{\mathbb Q}
\newcommand{\BZ}{\mathbb Z}
\newcommand{\BN}{\mathbb N}

\title{C\'odigo Nombre del Curso}
\author{Tu nombre aqu\'i}
\date{Fecha}

\begin{document}

\maketitle

\vspace*{-0.25in}
\centerline{Tu universidad aqu\'i}
\centerline{Tu regi\'on y pa\'is aqu\'i}
\centerline{\href{mailto:tucorreoaqui@pucp.edu.pe}{{\tt tucorreoaqui@pucp.edu.pe}}}
\vspace*{0.15in}

\begin{framed}
  Esta es una plantilla en \LaTeX. Util\'izala en tus listas de ejercicios, notas de clases y mucho m\'as.
\end{framed}

\begin{statement}{1}
    Este es el enunciado de un problema. Colocar el enunciado de un problema de un color diferente permite distinguir entre el problema y la soluci\'on.
\end{statement}

\begin{proof}
    Tipea tus soluciones en esta secci\'on. Usa las definiciones, lemas y ejemplos cuando sean necesarios.
    \begin{defn}
        Definimos $\exp(x)$ para $x \in \BR$ el valor de $$\sum_{i = 0}^\infty\frac{x^i}{i!}.$$
    \end{defn}
    Al igual que en la definici\'on anterior, usar ecuaciones en l\'ineas separadas cuando sea posible, esto hace tu problema m\'as le\'ible. Usa {\tt align*} para lista de igualdades:
    \begin{align*}
        0 &= 0 + 0 + 0 + 0 + \dots\\
        &= (1 - 1) + (1 - 1) + \dots \\
        &= 1 + (-1 + 1) + (-1 + 1) + \dots \\
        &= 1 + 0 + 0 + 0 \dots \\
        &= 1.
    \end{align*}
    Si necesitas listar cosas, usa {\tt enumerate} o {\tt itemize}. Por ejemplo:
    \begin{enumerate}
        \item Debo terminar la lista de ejercicios de \'Algebra Multilineal.
        \item Debo terminar la plantilla para Teor\'ia de Juegos.
        \item Debo volver a entrenar Programaci\'on Competitiva.
    \end{enumerate}
    Todas las secciones {\tt proof} terminan con el cuadrado que simboliza el hecho de que ha concluido la demostraci\'on.
\end{proof}

\begin{statement}{2}
    Si la soluci\'on de un problema es f\'acil de chequear si es correcta, ¿el problema debe ser f\'acil de resolver?
\end{statement}

\begin{proof}
    Buena suerte.
\end{proof}

\end{document}
